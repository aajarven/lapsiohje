\documentclass[12pt,twoside]{article}

\usepackage[ampersand]{easylist}
\usepackage[finnish]{babel}
\usepackage[utf8]{inputenc}
\usepackage{sectsty}
\usepackage{titlesec}
\usepackage{enumitem}
\usepackage{amssymb}
\usepackage{fancyhdr}
\usepackage{graphicx}
\usepackage{hyperref}
\usepackage[usenames, dvipsnames]{color}

\usepackage{sidecap}
\usepackage[font=sf]{caption}
\sidecaptionvpos{figure}{t}

\usepackage{geometry}
 \geometry{
 a4paper,
 total={170mm,240mm},
 left=20mm,
 top=25mm,
 footskip=18mm
 }

\fontfamily{lmss}\selectfont 
\pagestyle{fancy}
\fancyhf{}
\renewcommand{\headrulewidth}{0pt}
\fancyhead[LE,RO]{\thepage}
\lfoot{\fontfamily{lmss}\selectfont Creative Commons BY-NC-SA}
\rfoot{\fontfamily{lmss}\selectfont Alkuperäinen tekijä (2017) Anni Järvenpää}

\newenvironment{vaihetaso1}{%
    \LARGE
    \itemize
}{%
    \enditemize
}
\newenvironment{vaihetaso2}{%
    \Large
    \itemize
}{%
    \enditemize
}
\newenvironment{vaihetaso3}{%
    \large
    \itemize
}{%
    \enditemize
}

\setitemize{topsep=0pt,parsep=10pt}

\newcommand{\laatikkoyksi}{\raisebox{0ex}{$\square$}}
\renewcommand{\labelitemi}{\laatikkoyksi}
\newcommand{\laatikkokaksi}{\raisebox{0ex}{$\square$}}
\renewcommand{\labelitemii}{\laatikkokaksi}
\newcommand{\laatikkokolme}{\raisebox{0ex}{$\square$}}
\renewcommand{\labelitemiii}{\laatikkokolme}

\sectionfont{\fontfamily{lmss}\fontsize{35}{35}\selectfont}
\subsectionfont{\fontsize{25}{40}\fontfamily{lmss}\selectfont}
\titlespacing*{\section}{3em}{0em}{2em}
\titlespacing*{\subsection}{1.5em}{1em}{1em}

% vanhan scratchin nappuloiden kuvia
\newcommand{\valitseW}{\ensuremath{%
  \mathchoice{\includegraphics[height=2ex]{kuvat/valitse2.png}}
    {\includegraphics[height=2ex]{kuvat/valitse2.png}}
    {\includegraphics[height=1.5ex]{kuvat/valitse2.png}}
    {\includegraphics[height=1ex]{kuvat/valitse2.png}}
}}

\newcommand{\piirraW}{\ensuremath{%
  \mathchoice{\includegraphics[height=2ex]{kuvat/piirra2.png}}
    {\includegraphics[height=2ex]{kuvat/piirra2.png}}
    {\includegraphics[height=1.5ex]{kuvat/piirra2.png}}
    {\includegraphics[height=1ex]{kuvat/piirra2.png}}
}}

\newcommand{\lippuW}{\ensuremath{%
  \mathchoice{\includegraphics[height=2ex]{kuvat/lippu2.png}}
    {\includegraphics[height=2ex]{kuvat/lippu2.png}}
    {\includegraphics[height=1.5ex]{kuvat/lippu2.png}}
    {\includegraphics[height=1ex]{kuvat/lippu2.png}}
}}

\newcommand{\kopioiW}{\ensuremath{%
  \mathchoice{\includegraphics[height=2ex]{kuvat/kopioi2.png}}
    {\includegraphics[height=2ex]{kuvat/kopioi2.png}}
    {\includegraphics[height=1.5ex]{kuvat/kopioi2.png}}
    {\includegraphics[height=1ex]{kuvat/kopioi2.png}}
}}

\newcommand{\poistaW}{\ensuremath{%
  \mathchoice{\includegraphics[height=2ex]{kuvat/poista2.png}}
    {\includegraphics[height=2ex]{kuvat/poista2.png}}
    {\includegraphics[height=1.5ex]{kuvat/poista2.png}}
    {\includegraphics[height=1ex]{kuvat/poista2.png}}
}}

\newcommand{\kasvataW}{\ensuremath{%
  \mathchoice{\includegraphics[height=2ex]{kuvat/kasvata2.png}}
    {\includegraphics[height=2ex]{kuvat/kasvata2.png}}
    {\includegraphics[height=1.5ex]{kuvat/kasvata2.png}}
    {\includegraphics[height=1ex]{kuvat/kasvata2.png}}
}}

\newcommand{\pienennaW}{\ensuremath{%
  \mathchoice{\includegraphics[height=2ex]{kuvat/pienenna2.png}}
    {\includegraphics[height=2ex]{kuvat/pienenna2.png}}
    {\includegraphics[height=1.5ex]{kuvat/pienenna2.png}}
    {\includegraphics[height=1ex]{kuvat/pienenna2.png}}
}}

\newcommand{\jokasuuntaanW}{\ensuremath{%
  \mathchoice{\includegraphics[height=2ex]{kuvat/kiertotyyli-jokasuuntaan2.png}}
    {\includegraphics[height=2ex]{kuvat/kiertotyyli-jokasuuntaan2.png}}
    {\includegraphics[height=1.5ex]{kuvat/kiertotyyli-jokasuuntaan2.png}}
    {\includegraphics[height=1ex]{kuvat/kiertotyyli-jokasuuntaan2.png}}
}}

\newcommand{\vasenoikeaW}{\ensuremath{%
  \mathchoice{\includegraphics[height=2ex]{kuvat/kiertotyyli-vasenoikea2.png}}
    {\includegraphics[height=2ex]{kuvat/kiertotyyli-vasenoikea2.png}}
    {\includegraphics[height=1.5ex]{kuvat/kiertotyyli-vasenoikea2.png}}
    {\includegraphics[height=1ex]{kuvat/kiertotyyli-vasenoikea2.png}}
}}

\newcommand{\alakierraW}{\ensuremath{%
  \mathchoice{\includegraphics[height=2ex]{kuvat/kiertotyyli-alakierra2.png}}
    {\includegraphics[height=2ex]{kuvat/kiertotyyli-alakierra2.png}}
    {\includegraphics[height=1.5ex]{kuvat/kiertotyyli-alakierra2.png}}
    {\includegraphics[height=1ex]{kuvat/kiertotyyli-alakierra2.png}}
}}
% Uuden scartchin värien määrittelyt
\definecolor{liike}{RGB}{76, 151, 255}
\definecolor{ulkonako}{RGB}{153, 102, 255}
\definecolor{aani}{RGB}{214, 92, 214}
\definecolor{laajennus}{RGB}{15, 189, 140}
\definecolor{muuttujat}{RGB}{255, 140, 26}
\definecolor{tapahtumat}{RGB}{255, 213, 0}
\definecolor{ohjaus}{RGB}{255, 171, 25}
\definecolor{tuntoaisti}{RGB}{76, 191, 230}
\definecolor{toiminnot}{RGB}{64, 191, 74}
\definecolor{lohkot}{RGB}{255, 102, 128}

% Scratch 2.0 -värien määrittelyt
\definecolor{liike2}{RGB}{74, 108, 212}
\definecolor{ulkonako2}{RGB}{138, 85, 215}
\definecolor{aani2}{RGB}{187, 66, 195}
\definecolor{kyna2}{RGB}{14, 154, 108}
\definecolor{tieto2}{RGB}{238, 125, 22}
\definecolor{tapahtumat2}{RGB}{200, 131, 48}
\definecolor{ohjaus2}{RGB}{225, 169, 26}
\definecolor{tuntoaisti2}{RGB}{44, 165, 226}
\definecolor{toiminnot2}{RGB}{92, 183, 18}
\definecolor{lohkot2}{RGB}{99, 45, 153}


\begin{document}
\fontfamily{lmss}\selectfont
\section*{Pelin nimi}

\begin{SCfigure}[][h]
  \centering
  \captionsetup{labelformat=empty}
  \caption{Lyhyt kuvaus pelistä, mitä siinä tehdään ja mihin pyritään. \href{https://google.fi}{Linkki mallipeliin}, tekijän nimi ja pelin nimi näkyviin että tulosteenkin perusteella on mahdollista löytää.}
  \includegraphics[width=0.7\textwidth]{kuvat/esimerkki.png}
\end{SCfigure}

\begin{vaihetaso1}
	\item Ensimmäinen ohje
	\begin{vaihetaso2}
		\item Yhden lauseen kuvaus
		\item Tällä on alakohta
		\begin{vaihetaso3}
			\item Turhan syvä alakohta
		\end{vaihetaso3}
		\item Pitkähkö kuvaus siitä, mitä nyt pitää tehdä. Monimutkaiset ohjelmat ovat monimutkaisia. Onneksi lapset ovat fiksuja.
	\end{vaihetaso2}
	\item Toinen vaihe
	\begin{vaihetaso2}
		\item Laatikoita pukkaa
		\item Tällä on alakohta
		\item Lorem ipsum
	\end{vaihetaso2}
	\item Kolmas vaihe
	\item Neljäs vaihe
\end{vaihetaso1}

\subsection*{Esimerkkejä värien käytöistä}
\begin{vaihetaso1}
	\item \textcolor{liike}{Liikunta} on tärkeämpää kuin \textcolor{ulkonako}{ulkonäkö}. 
	\begin{vaihetaso2}
		\item Välillä voi käyttää \textcolor{aani}{ääntä}.
		\item Oispa \textcolor{muuttujat}{muuttujia}.
		\begin{vaihetaso3}
			\item Muita vaihtoehtoja ovat \textcolor{ohjaus}{ohjaus} ja \textcolor{tuntoaisti}{tuntoaisti}.
		\end{vaihetaso3}
	\end{vaihetaso2}
	\item \textcolor{lohkot}{Lohkot} \textcolor{toiminnot}{toimii}.
	\item \textcolor{laajennus}{Kynä} on uudessa Scratchissa laajennus eli samalla värillä kuin esimerkiksi \textcolor{laajennus}{musiikki-} ja \textcolor{laajennus}{käännöspalikat}.
	\item Myös vanhan scratchin värejä voi tarvittaessa käyttää. Ne on merkattu numerolla 2 nimen lopussa, esim \textcolor{lohkot2}{vanhat lohkot}.
\end{vaihetaso1}

\subsection*{Ikoneita}
\begin{vaihetaso1}
	\item Wanha scratch (merkattu kirjaimella ''W'' nimen jälkeen):
	\begin{vaihetaso2}
		\item Grafiikoita voi piirtää $\piirraW$ tai valita valmiina $\valitseW$. Vihreästä lipusta $\lippuW$ starttaa.
		\item Yläpalkissa on myös namiskoja, esim $\kopioiW$ kopiointi, $\poistaW$ poisto, $\kasvataW$ kasvatus ja $\pienennaW$ pienennys.
		\item Hahmojen kiertotyyliksi voi asettaa $\jokasuuntaanW$, $\vasenoikeaW$ tai $\alakierraW$.
	\end{vaihetaso2}
\end{vaihetaso1}

\subsection*{Väliotsikko}

\begin{vaihetaso1}
	\item Ensimmäinen ohje
	\begin{vaihetaso2}
		\item Yhden lauseen kuvaus
		\item Tällä on alakohta
		\begin{vaihetaso3}
			\item Turhan syvä alakohta
		\end{vaihetaso3}
		\item Pitkähkö kuvaus siitä, mitä nyt pitää tehdä. Monimutkaiset ohjelmat ovat monimutkaisia. Onneksi lapset ovat fiksuja.
	\end{vaihetaso2}
	\item Toinen vaihe
	\item Vielä yksi
	\begin{vaihetaso2}
		\item Laatikoita pukkaa
		\item Tällä on alakohta
		\item Lorem ipsum
	\end{vaihetaso2}
	\item Kolmas vaihe
	\item Neljäs vaihe
\end{vaihetaso1}

% halutessaan voi alemman %-merkin poistamalla pakottaa laajennusideat omalle sivulleen
%\newpage
\subsection*{Laajennusideoita}
\begin{itemize}
	\item[-] Muita hauskoja juttuja, joita peliin voisi halutessaan toteuttaa
	\item[-] Voi olla myös alakohtia:
	\begin{itemize}
		\item[-] Ensimmäinen alakohta
		\item[-] Toinen alakohta
		\item[-] Kolmas alakohta
	\end{itemize} 
	\item[-] Vielä tavallinen idea perään
\end{itemize}

\end{document}